% **************************************************************************** %
%                                                                              %
%                                                     ::::::::  :::::::::      %
%    sec_mteorico.tex                                :+:    :+: :+:    :+:     %
%                                                    +:+        +:+    +:+     %
%    By: A. Campo <andoitzcp@gmail.com>              +#+        +#++:++#+      %
%                                                    +#+        +#+            %
%    Created: 2022/11/22 06:21:10 by A. Campo        #+#    #+# #+#            %
%    Updated: 2022/12/07 00:26:53 by A. Campo         ########  ###            %
%                                                                              %
% **************************************************************************** %

\section{MARCO TEÓRICO}

\subsection{SIMULACIÓN DE EVENTOS DISCRETOS}

Con el avance tecnológico experimentado desde los inicios de la computación,
nuevas maneras para afrontar la resolución de problemas han surgido,
que un siglo antes, habrían sido descartadas por su falta de viabilidad.
La capacidad de calculo y el diseño de inteligencias artificiales,
a proporcionado las herramientas necesarias
para solucionar aquellos problemas, que con métodos clásicos,
definitivamente demasiado complejos para modelar.
La simulación, hoy en día, es una de las tres metodologías consolidadas,
en el ámbito científico e ingenieril,
para la resolución de problemas \citep{banks1998handbook}.
Siendo esta metodología descrita como "la técnica del ultimo recurso"
por el autor \citep{garzia1986discrete},
debido a la intensa demanda computacional
que requería en el momento de su publicación,
el transcurso de los años ha mermado esta desventaja.

Diversos autores describen la simulación de la siguiente manera

\begin{itemize}
	\item ``Una simulación es
	el establecimiento de un modelo lógico-matemático de un sistema
	y la manipulación experimental de este en una computadora digital
	\citep{pritsker1974gasp}''.   
	\item ``La simulación es el proceso de diseñar
		el modelo de un sistema real y realizar experimentos
		con este modelo con el propósito de,
		o bien,
		entender el comportamiento del sistema,
		o evaluar distintas estrategias
		(dentro de los limites impuestos por un criterio
		o conjunto de criterios)
		para la operación del sistema'' \citep{shannon1976systems}.
	\item ``Una simulación es la imitación de el modo de operación
		de un proceso real o sistema durante el transcurso del tiempo''
		\citep{banks1999introduction}.
\end{itemize}

Como se puede observar, el modelo,
es un termino recurrente en el ámbito de la simulación.
Banks sostiene que el modelo es la representación del sistema simulado
\citep{banks1998handbook}.
El autor matiza que la virtud de un modelo radica en su complejidad,
siendo necesario el balance entre una representación ajustada,
sin complicar en exceso el modelo.
Por tanto, aquellos factores que no influyan lo suficiente
en los resultados de la simulación, deberían ser eliminados,
ya que únicamente extienden el proceso de desarrollo.

El modelo, por tanto, constituye la base de la simulación,
define las variables, y criterios para las decisiones tomadas en la simulación.
Banks señala, la distinción entre un modelo matemático convencional
y el modelo de una simulación \citep{banks1998handbook}.
Los modelos matemáticos y estadísticos
suelen representar las variables de manera explicita,
relacionando variables independientes y variables dependientes
para obtener un resultado.
En el caso de una DES, el modelo utilizado se enfoca en
la representación de sus componentes internos y sus interacciones..

El funcionamiento de la DES, se basa en registrar los cambios
que van ocurriendo en el estado del sistema durante el transcurso del tiempo.
En el intervalo entre alternaciones de estado,
ocurren el concepto de eventos, propios de esta rama de de simulación.
Varga \citep{varga2001discrete} describe estos eventos como instantáneos,
siendo la duración de los mismos nula.
Durante el evento no ocurre nada especial,
salvo la alteración del estado del sistema.
Una sencilla representación de este concepto
puede ser observada el la figura \ref{fig:2_basic_flow_des},
donde el sistema representado es un coche.
El coche posee 3 estados, los cuales se van sucediendo,
durante el transcurso del tiempo.

\begin{figure}[h]
	\begin{center}
		\includegraphics{2_basic_flow_des}
	\end{center}
	\caption{Representación del cambio de estado en una DES,
	los rectángulos representan el estado del sistema,
	mientras que las flechas indican el suceso de un evento instantáneo.}
	\label{fig:2_basic_flow_des}
\end{figure}

\subsubsection{Terminologia y funcionamiento de una DES}\label{TF_DES}

Para describir detalladamente este tipo de simulaciones,
es necesario definir previamente algunos conceptos.
Los siguientes conceptos
son la sintesis de la informacion redactada por \citep{banks1998handbook}
aplicados en el entorno de Python y su libreria Simpy.

\begin{itemize}
	\item Las variables del estado del sistema proporcionan informacion
		acerca de el comportamiento de los procesos
		que se llevan a cabo durante la simulacion.
		Determinar las variables de estado que el programa debe retornar
		depende de el objetivo del proyecto.
		El programa debera ser diseñado entorno a ellas,
		puesto que,seran usadas
		para el posterior calculo de indicadores del rendimiento.
		Un ejemplo, es el momento de inicio y fin de un subproceso.

	\item Las entidades so los objetos representados por el modelo.
		Se distinguen dos tipos de entidades,
		las dinamicas y las estaticas.
		Las primeras, avanzan en el sistema
		a traves de los procesos simulados, 
		mientras que las segundas, dan servicio a las primeras
		permaneciendo ocupas en el proceso.
		Considerando una gasolinera como sistema,
		la entidad dinamica seria un coche repostando combustible,
		mientras que la estatica, seria la estacion de repostaje.
		Estos objetos pueden poseer atributos,
		como el consumo de combustible o el tamaño de su deposito.
		Dependiendo de el objetivo de la simulacion,
		algunos atributos seran relevantes en diseño del modelo.
	\item Un recurso es una entidad
		que provee de servicio a una entidad dinamica.
		Los recursos pueden servir a multiples entidades simultaneamente.
		De manera similar, una entidad puede
		solicitar varios tipos de recursos en distintas cantidades.
		Si la demanda de una entidad no se satisface,
		esta entrara en una cola,
		a la espera de la liberacion del recurso.
		Cuando la entidad tome el recurso,
		lo mantendra ocupado durante la duracion del proceso.
		Los recursos pueden tener varios estados,
		como disponible, ocupado, averiado, en mantenimiento...
	\item Una actividad es un periodo de tiempo,
		el cual es conocido previo a su inicio.
		Su duracion puede ser determinada de varias maneras;
		mediante una constante, una distrubucion estadistica,
		un resuldo de una ecuacion, un archivo o una decision logica.
		El conocimiento de su duracion,
		supone saber cuando un recurso sera liberado,
		lo que facilita el trabajo de computacion.
	\item Una demora es un periodo de espera indefinido
		antes del comienzo de una actividad.
		Este depende de la ejecucion
		de el resto de actividades prioritarias,
		por lo que el final de la espera no se puede determinar
		mientras el recurso siga ocupado.
	\item Un evento ocurre al principio y al final de una actividad o demora.
		Marcan la transicion de el sistema de un estado a otro.
\end{itemize}

Considerando todos estos conceptos,
las entidades del sistema compiten entre ellas por el uso de los recursos,
si estos no se encuentran disponibles, quedan a la espera de poder solicitarlos.
Tanto las actividades como las demoras reclaman estas entidades
durante los intervalos entre eventos,
siendo liberadas para el siguiente procesouna vez finalizado el periodo.
Una DES, puede describirse como el avance de una entidad o conjunto de entidades
a traves  de las actividades que modelan el proceso real.
Este avance es accionado
mediante el trancurso de tiempo virtual dentro de la simulacion.

\subsubsection{Exposición de un caso ficticio usando una DES}

Con el objetivo de ilustrar los conceptos descritos en la sección \ref{TF_DES}
 demostrar la capacidad de resolución de problemas de la DES,
 se ha procedido a emular un caso práctico.
 omado como referencia el modelo descrito en la figura
 \ref{fig:diagrama_basico_DES}, se ha redactado la siguiente problemática.

 Supongase una jornada laboral
 que comienza desde el amanecer y que finaliza al atardecer del mismo día.
 Durante este periodo, varios individuos deciden
 tomar el coche para desplazarse a distintos lugares.
 La mayoria seran empleados
 que se desplazara a su lugar de trabajo en horarios similares.
 mientras que los demás serán conductores ocasionales
 que usen su vehículo a lo largo del día repartidos de manera equitativa.
 De esta población una fracción deseará repostar
 en la gasolinera más cercana a el parking donde estacionaran sus vehículos.

 El comportamiento del sistema descrito,
 puede ser emulado mediante una DES aplicando las variables expuestas
 en la tabla y el diagrama de la figura \ref{fig:simulation_example_diagram}.

\begin{figure}
	\begin{center}
		\includegraphics{simulation_example_diagram}
	\end{center}
		\caption{Modelo de ejemplo para demostrar el funcionamiento de una DES.}
		\label{fig:simulation_example_diagram}
\end{figure}

\subsubsection{Análisis DAFO}

\subsubsection{Etapas de desarrollo de una DES}

\begin{figure}
	\begin{center}
		\includegraphics{simulation_diagram}
	\end{center}
	\caption{Esquema del proceso de desarrollo de una DES.}
	\label{fig:simulation_diagram}
\end{figure}
