% **************************************************************************** %
%                                                                              %
%                                                     ::::::::  :::::::::      %
%    sec_mteorico.tex                                :+:    :+: :+:    :+:     %
%                                                    +:+        +:+    +:+     %
%    By: A. Campo <andoitzcp@gmail.com>              +#+        +#++:++#+      %
%                                                    +#+        +#+            %
%    Created: 2022/11/22 06:21:10 by A. Campo        #+#    #+# #+#            %
%    Updated: 2022/11/22 08:26:43 by A. Campo         ########  ###            %
%                                                                              %
% **************************************************************************** %

\section{MARCO TEORICO}

\subsection{SIMULACION DE EVENTOS DISCRETOS}

Con el avance tecnologico experimentado desde los inicios de la computacion,
nuevas maneras para afrontar la resolucion de problemas han surgido,
que un siglo antes, habrian sido descartadas por su falta de viabilidad.
La capacidad de calculo y el dise;o de inteligencias artifiiciales,
ha proporcionado las herramientas necesarias
para solucionar aquellos problemas, que con metodos clasicos,
definitavamente demasiado complejos para modelar.
La simulacion, hoy en dia, es una de las tres metodologias consolidadas,
en el ambito cientifico e ingenieril,
para la resolucion de problemas \citep{banks1998handbook}.
Siendo esta metodologia descrita como "la tecnica del ultimo recurso"
por el autor \citep{garzia1986discrete},
debido a la intensa demanda computacional que requeria en aquellos tiempos,
el transcurso de los años ha mermado esta desventaja.

Diversos autores describen la simulacion de la siguiente manera

\begin{itemize}
	\item ``Una simulacion es
	el establecimiento de un modelo logico-matematico de un sistema
	y la manipulacion experimental de este en una computadora digital
	\citep{pritsker1974gasp}''.   
\end{itemize}
